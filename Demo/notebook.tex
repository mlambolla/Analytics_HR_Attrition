
% Default to the notebook output style

    


% Inherit from the specified cell style.




    
\documentclass[11pt]{article}

    
    
    \usepackage[T1]{fontenc}
    % Nicer default font (+ math font) than Computer Modern for most use cases
    \usepackage{mathpazo}

    % Basic figure setup, for now with no caption control since it's done
    % automatically by Pandoc (which extracts ![](path) syntax from Markdown).
    \usepackage{graphicx}
    % We will generate all images so they have a width \maxwidth. This means
    % that they will get their normal width if they fit onto the page, but
    % are scaled down if they would overflow the margins.
    \makeatletter
    \def\maxwidth{\ifdim\Gin@nat@width>\linewidth\linewidth
    \else\Gin@nat@width\fi}
    \makeatother
    \let\Oldincludegraphics\includegraphics
    % Set max figure width to be 80% of text width, for now hardcoded.
    \renewcommand{\includegraphics}[1]{\Oldincludegraphics[width=.8\maxwidth]{#1}}
    % Ensure that by default, figures have no caption (until we provide a
    % proper Figure object with a Caption API and a way to capture that
    % in the conversion process - todo).
    \usepackage{caption}
    \DeclareCaptionLabelFormat{nolabel}{}
    \captionsetup{labelformat=nolabel}

    \usepackage{adjustbox} % Used to constrain images to a maximum size 
    \usepackage{xcolor} % Allow colors to be defined
    \usepackage{enumerate} % Needed for markdown enumerations to work
    \usepackage{geometry} % Used to adjust the document margins
    \usepackage{amsmath} % Equations
    \usepackage{amssymb} % Equations
    \usepackage{textcomp} % defines textquotesingle
    % Hack from http://tex.stackexchange.com/a/47451/13684:
    \AtBeginDocument{%
        \def\PYZsq{\textquotesingle}% Upright quotes in Pygmentized code
    }
    \usepackage{upquote} % Upright quotes for verbatim code
    \usepackage{eurosym} % defines \euro
    \usepackage[mathletters]{ucs} % Extended unicode (utf-8) support
    \usepackage[utf8x]{inputenc} % Allow utf-8 characters in the tex document
    \usepackage{fancyvrb} % verbatim replacement that allows latex
    \usepackage{grffile} % extends the file name processing of package graphics 
                         % to support a larger range 
    % The hyperref package gives us a pdf with properly built
    % internal navigation ('pdf bookmarks' for the table of contents,
    % internal cross-reference links, web links for URLs, etc.)
    \usepackage{hyperref}
    \usepackage{longtable} % longtable support required by pandoc >1.10
    \usepackage{booktabs}  % table support for pandoc > 1.12.2
    \usepackage[inline]{enumitem} % IRkernel/repr support (it uses the enumerate* environment)
    \usepackage[normalem]{ulem} % ulem is needed to support strikethroughs (\sout)
                                % normalem makes italics be italics, not underlines
    

    
    
    % Colors for the hyperref package
    \definecolor{urlcolor}{rgb}{0,.145,.698}
    \definecolor{linkcolor}{rgb}{.71,0.21,0.01}
    \definecolor{citecolor}{rgb}{.12,.54,.11}

    % ANSI colors
    \definecolor{ansi-black}{HTML}{3E424D}
    \definecolor{ansi-black-intense}{HTML}{282C36}
    \definecolor{ansi-red}{HTML}{E75C58}
    \definecolor{ansi-red-intense}{HTML}{B22B31}
    \definecolor{ansi-green}{HTML}{00A250}
    \definecolor{ansi-green-intense}{HTML}{007427}
    \definecolor{ansi-yellow}{HTML}{DDB62B}
    \definecolor{ansi-yellow-intense}{HTML}{B27D12}
    \definecolor{ansi-blue}{HTML}{208FFB}
    \definecolor{ansi-blue-intense}{HTML}{0065CA}
    \definecolor{ansi-magenta}{HTML}{D160C4}
    \definecolor{ansi-magenta-intense}{HTML}{A03196}
    \definecolor{ansi-cyan}{HTML}{60C6C8}
    \definecolor{ansi-cyan-intense}{HTML}{258F8F}
    \definecolor{ansi-white}{HTML}{C5C1B4}
    \definecolor{ansi-white-intense}{HTML}{A1A6B2}

    % commands and environments needed by pandoc snippets
    % extracted from the output of `pandoc -s`
    \providecommand{\tightlist}{%
      \setlength{\itemsep}{0pt}\setlength{\parskip}{0pt}}
    \DefineVerbatimEnvironment{Highlighting}{Verbatim}{commandchars=\\\{\}}
    % Add ',fontsize=\small' for more characters per line
    \newenvironment{Shaded}{}{}
    \newcommand{\KeywordTok}[1]{\textcolor[rgb]{0.00,0.44,0.13}{\textbf{{#1}}}}
    \newcommand{\DataTypeTok}[1]{\textcolor[rgb]{0.56,0.13,0.00}{{#1}}}
    \newcommand{\DecValTok}[1]{\textcolor[rgb]{0.25,0.63,0.44}{{#1}}}
    \newcommand{\BaseNTok}[1]{\textcolor[rgb]{0.25,0.63,0.44}{{#1}}}
    \newcommand{\FloatTok}[1]{\textcolor[rgb]{0.25,0.63,0.44}{{#1}}}
    \newcommand{\CharTok}[1]{\textcolor[rgb]{0.25,0.44,0.63}{{#1}}}
    \newcommand{\StringTok}[1]{\textcolor[rgb]{0.25,0.44,0.63}{{#1}}}
    \newcommand{\CommentTok}[1]{\textcolor[rgb]{0.38,0.63,0.69}{\textit{{#1}}}}
    \newcommand{\OtherTok}[1]{\textcolor[rgb]{0.00,0.44,0.13}{{#1}}}
    \newcommand{\AlertTok}[1]{\textcolor[rgb]{1.00,0.00,0.00}{\textbf{{#1}}}}
    \newcommand{\FunctionTok}[1]{\textcolor[rgb]{0.02,0.16,0.49}{{#1}}}
    \newcommand{\RegionMarkerTok}[1]{{#1}}
    \newcommand{\ErrorTok}[1]{\textcolor[rgb]{1.00,0.00,0.00}{\textbf{{#1}}}}
    \newcommand{\NormalTok}[1]{{#1}}
    
    % Additional commands for more recent versions of Pandoc
    \newcommand{\ConstantTok}[1]{\textcolor[rgb]{0.53,0.00,0.00}{{#1}}}
    \newcommand{\SpecialCharTok}[1]{\textcolor[rgb]{0.25,0.44,0.63}{{#1}}}
    \newcommand{\VerbatimStringTok}[1]{\textcolor[rgb]{0.25,0.44,0.63}{{#1}}}
    \newcommand{\SpecialStringTok}[1]{\textcolor[rgb]{0.73,0.40,0.53}{{#1}}}
    \newcommand{\ImportTok}[1]{{#1}}
    \newcommand{\DocumentationTok}[1]{\textcolor[rgb]{0.73,0.13,0.13}{\textit{{#1}}}}
    \newcommand{\AnnotationTok}[1]{\textcolor[rgb]{0.38,0.63,0.69}{\textbf{\textit{{#1}}}}}
    \newcommand{\CommentVarTok}[1]{\textcolor[rgb]{0.38,0.63,0.69}{\textbf{\textit{{#1}}}}}
    \newcommand{\VariableTok}[1]{\textcolor[rgb]{0.10,0.09,0.49}{{#1}}}
    \newcommand{\ControlFlowTok}[1]{\textcolor[rgb]{0.00,0.44,0.13}{\textbf{{#1}}}}
    \newcommand{\OperatorTok}[1]{\textcolor[rgb]{0.40,0.40,0.40}{{#1}}}
    \newcommand{\BuiltInTok}[1]{{#1}}
    \newcommand{\ExtensionTok}[1]{{#1}}
    \newcommand{\PreprocessorTok}[1]{\textcolor[rgb]{0.74,0.48,0.00}{{#1}}}
    \newcommand{\AttributeTok}[1]{\textcolor[rgb]{0.49,0.56,0.16}{{#1}}}
    \newcommand{\InformationTok}[1]{\textcolor[rgb]{0.38,0.63,0.69}{\textbf{\textit{{#1}}}}}
    \newcommand{\WarningTok}[1]{\textcolor[rgb]{0.38,0.63,0.69}{\textbf{\textit{{#1}}}}}
    
    
    % Define a nice break command that doesn't care if a line doesn't already
    % exist.
    \def\br{\hspace*{\fill} \\* }
    % Math Jax compatability definitions
    \def\gt{>}
    \def\lt{<}
    % Document parameters
    \title{Demo - Attrition}
    
    
    

    % Pygments definitions
    
\makeatletter
\def\PY@reset{\let\PY@it=\relax \let\PY@bf=\relax%
    \let\PY@ul=\relax \let\PY@tc=\relax%
    \let\PY@bc=\relax \let\PY@ff=\relax}
\def\PY@tok#1{\csname PY@tok@#1\endcsname}
\def\PY@toks#1+{\ifx\relax#1\empty\else%
    \PY@tok{#1}\expandafter\PY@toks\fi}
\def\PY@do#1{\PY@bc{\PY@tc{\PY@ul{%
    \PY@it{\PY@bf{\PY@ff{#1}}}}}}}
\def\PY#1#2{\PY@reset\PY@toks#1+\relax+\PY@do{#2}}

\expandafter\def\csname PY@tok@w\endcsname{\def\PY@tc##1{\textcolor[rgb]{0.73,0.73,0.73}{##1}}}
\expandafter\def\csname PY@tok@c\endcsname{\let\PY@it=\textit\def\PY@tc##1{\textcolor[rgb]{0.25,0.50,0.50}{##1}}}
\expandafter\def\csname PY@tok@cp\endcsname{\def\PY@tc##1{\textcolor[rgb]{0.74,0.48,0.00}{##1}}}
\expandafter\def\csname PY@tok@k\endcsname{\let\PY@bf=\textbf\def\PY@tc##1{\textcolor[rgb]{0.00,0.50,0.00}{##1}}}
\expandafter\def\csname PY@tok@kp\endcsname{\def\PY@tc##1{\textcolor[rgb]{0.00,0.50,0.00}{##1}}}
\expandafter\def\csname PY@tok@kt\endcsname{\def\PY@tc##1{\textcolor[rgb]{0.69,0.00,0.25}{##1}}}
\expandafter\def\csname PY@tok@o\endcsname{\def\PY@tc##1{\textcolor[rgb]{0.40,0.40,0.40}{##1}}}
\expandafter\def\csname PY@tok@ow\endcsname{\let\PY@bf=\textbf\def\PY@tc##1{\textcolor[rgb]{0.67,0.13,1.00}{##1}}}
\expandafter\def\csname PY@tok@nb\endcsname{\def\PY@tc##1{\textcolor[rgb]{0.00,0.50,0.00}{##1}}}
\expandafter\def\csname PY@tok@nf\endcsname{\def\PY@tc##1{\textcolor[rgb]{0.00,0.00,1.00}{##1}}}
\expandafter\def\csname PY@tok@nc\endcsname{\let\PY@bf=\textbf\def\PY@tc##1{\textcolor[rgb]{0.00,0.00,1.00}{##1}}}
\expandafter\def\csname PY@tok@nn\endcsname{\let\PY@bf=\textbf\def\PY@tc##1{\textcolor[rgb]{0.00,0.00,1.00}{##1}}}
\expandafter\def\csname PY@tok@ne\endcsname{\let\PY@bf=\textbf\def\PY@tc##1{\textcolor[rgb]{0.82,0.25,0.23}{##1}}}
\expandafter\def\csname PY@tok@nv\endcsname{\def\PY@tc##1{\textcolor[rgb]{0.10,0.09,0.49}{##1}}}
\expandafter\def\csname PY@tok@no\endcsname{\def\PY@tc##1{\textcolor[rgb]{0.53,0.00,0.00}{##1}}}
\expandafter\def\csname PY@tok@nl\endcsname{\def\PY@tc##1{\textcolor[rgb]{0.63,0.63,0.00}{##1}}}
\expandafter\def\csname PY@tok@ni\endcsname{\let\PY@bf=\textbf\def\PY@tc##1{\textcolor[rgb]{0.60,0.60,0.60}{##1}}}
\expandafter\def\csname PY@tok@na\endcsname{\def\PY@tc##1{\textcolor[rgb]{0.49,0.56,0.16}{##1}}}
\expandafter\def\csname PY@tok@nt\endcsname{\let\PY@bf=\textbf\def\PY@tc##1{\textcolor[rgb]{0.00,0.50,0.00}{##1}}}
\expandafter\def\csname PY@tok@nd\endcsname{\def\PY@tc##1{\textcolor[rgb]{0.67,0.13,1.00}{##1}}}
\expandafter\def\csname PY@tok@s\endcsname{\def\PY@tc##1{\textcolor[rgb]{0.73,0.13,0.13}{##1}}}
\expandafter\def\csname PY@tok@sd\endcsname{\let\PY@it=\textit\def\PY@tc##1{\textcolor[rgb]{0.73,0.13,0.13}{##1}}}
\expandafter\def\csname PY@tok@si\endcsname{\let\PY@bf=\textbf\def\PY@tc##1{\textcolor[rgb]{0.73,0.40,0.53}{##1}}}
\expandafter\def\csname PY@tok@se\endcsname{\let\PY@bf=\textbf\def\PY@tc##1{\textcolor[rgb]{0.73,0.40,0.13}{##1}}}
\expandafter\def\csname PY@tok@sr\endcsname{\def\PY@tc##1{\textcolor[rgb]{0.73,0.40,0.53}{##1}}}
\expandafter\def\csname PY@tok@ss\endcsname{\def\PY@tc##1{\textcolor[rgb]{0.10,0.09,0.49}{##1}}}
\expandafter\def\csname PY@tok@sx\endcsname{\def\PY@tc##1{\textcolor[rgb]{0.00,0.50,0.00}{##1}}}
\expandafter\def\csname PY@tok@m\endcsname{\def\PY@tc##1{\textcolor[rgb]{0.40,0.40,0.40}{##1}}}
\expandafter\def\csname PY@tok@gh\endcsname{\let\PY@bf=\textbf\def\PY@tc##1{\textcolor[rgb]{0.00,0.00,0.50}{##1}}}
\expandafter\def\csname PY@tok@gu\endcsname{\let\PY@bf=\textbf\def\PY@tc##1{\textcolor[rgb]{0.50,0.00,0.50}{##1}}}
\expandafter\def\csname PY@tok@gd\endcsname{\def\PY@tc##1{\textcolor[rgb]{0.63,0.00,0.00}{##1}}}
\expandafter\def\csname PY@tok@gi\endcsname{\def\PY@tc##1{\textcolor[rgb]{0.00,0.63,0.00}{##1}}}
\expandafter\def\csname PY@tok@gr\endcsname{\def\PY@tc##1{\textcolor[rgb]{1.00,0.00,0.00}{##1}}}
\expandafter\def\csname PY@tok@ge\endcsname{\let\PY@it=\textit}
\expandafter\def\csname PY@tok@gs\endcsname{\let\PY@bf=\textbf}
\expandafter\def\csname PY@tok@gp\endcsname{\let\PY@bf=\textbf\def\PY@tc##1{\textcolor[rgb]{0.00,0.00,0.50}{##1}}}
\expandafter\def\csname PY@tok@go\endcsname{\def\PY@tc##1{\textcolor[rgb]{0.53,0.53,0.53}{##1}}}
\expandafter\def\csname PY@tok@gt\endcsname{\def\PY@tc##1{\textcolor[rgb]{0.00,0.27,0.87}{##1}}}
\expandafter\def\csname PY@tok@err\endcsname{\def\PY@bc##1{\setlength{\fboxsep}{0pt}\fcolorbox[rgb]{1.00,0.00,0.00}{1,1,1}{\strut ##1}}}
\expandafter\def\csname PY@tok@kc\endcsname{\let\PY@bf=\textbf\def\PY@tc##1{\textcolor[rgb]{0.00,0.50,0.00}{##1}}}
\expandafter\def\csname PY@tok@kd\endcsname{\let\PY@bf=\textbf\def\PY@tc##1{\textcolor[rgb]{0.00,0.50,0.00}{##1}}}
\expandafter\def\csname PY@tok@kn\endcsname{\let\PY@bf=\textbf\def\PY@tc##1{\textcolor[rgb]{0.00,0.50,0.00}{##1}}}
\expandafter\def\csname PY@tok@kr\endcsname{\let\PY@bf=\textbf\def\PY@tc##1{\textcolor[rgb]{0.00,0.50,0.00}{##1}}}
\expandafter\def\csname PY@tok@bp\endcsname{\def\PY@tc##1{\textcolor[rgb]{0.00,0.50,0.00}{##1}}}
\expandafter\def\csname PY@tok@fm\endcsname{\def\PY@tc##1{\textcolor[rgb]{0.00,0.00,1.00}{##1}}}
\expandafter\def\csname PY@tok@vc\endcsname{\def\PY@tc##1{\textcolor[rgb]{0.10,0.09,0.49}{##1}}}
\expandafter\def\csname PY@tok@vg\endcsname{\def\PY@tc##1{\textcolor[rgb]{0.10,0.09,0.49}{##1}}}
\expandafter\def\csname PY@tok@vi\endcsname{\def\PY@tc##1{\textcolor[rgb]{0.10,0.09,0.49}{##1}}}
\expandafter\def\csname PY@tok@vm\endcsname{\def\PY@tc##1{\textcolor[rgb]{0.10,0.09,0.49}{##1}}}
\expandafter\def\csname PY@tok@sa\endcsname{\def\PY@tc##1{\textcolor[rgb]{0.73,0.13,0.13}{##1}}}
\expandafter\def\csname PY@tok@sb\endcsname{\def\PY@tc##1{\textcolor[rgb]{0.73,0.13,0.13}{##1}}}
\expandafter\def\csname PY@tok@sc\endcsname{\def\PY@tc##1{\textcolor[rgb]{0.73,0.13,0.13}{##1}}}
\expandafter\def\csname PY@tok@dl\endcsname{\def\PY@tc##1{\textcolor[rgb]{0.73,0.13,0.13}{##1}}}
\expandafter\def\csname PY@tok@s2\endcsname{\def\PY@tc##1{\textcolor[rgb]{0.73,0.13,0.13}{##1}}}
\expandafter\def\csname PY@tok@sh\endcsname{\def\PY@tc##1{\textcolor[rgb]{0.73,0.13,0.13}{##1}}}
\expandafter\def\csname PY@tok@s1\endcsname{\def\PY@tc##1{\textcolor[rgb]{0.73,0.13,0.13}{##1}}}
\expandafter\def\csname PY@tok@mb\endcsname{\def\PY@tc##1{\textcolor[rgb]{0.40,0.40,0.40}{##1}}}
\expandafter\def\csname PY@tok@mf\endcsname{\def\PY@tc##1{\textcolor[rgb]{0.40,0.40,0.40}{##1}}}
\expandafter\def\csname PY@tok@mh\endcsname{\def\PY@tc##1{\textcolor[rgb]{0.40,0.40,0.40}{##1}}}
\expandafter\def\csname PY@tok@mi\endcsname{\def\PY@tc##1{\textcolor[rgb]{0.40,0.40,0.40}{##1}}}
\expandafter\def\csname PY@tok@il\endcsname{\def\PY@tc##1{\textcolor[rgb]{0.40,0.40,0.40}{##1}}}
\expandafter\def\csname PY@tok@mo\endcsname{\def\PY@tc##1{\textcolor[rgb]{0.40,0.40,0.40}{##1}}}
\expandafter\def\csname PY@tok@ch\endcsname{\let\PY@it=\textit\def\PY@tc##1{\textcolor[rgb]{0.25,0.50,0.50}{##1}}}
\expandafter\def\csname PY@tok@cm\endcsname{\let\PY@it=\textit\def\PY@tc##1{\textcolor[rgb]{0.25,0.50,0.50}{##1}}}
\expandafter\def\csname PY@tok@cpf\endcsname{\let\PY@it=\textit\def\PY@tc##1{\textcolor[rgb]{0.25,0.50,0.50}{##1}}}
\expandafter\def\csname PY@tok@c1\endcsname{\let\PY@it=\textit\def\PY@tc##1{\textcolor[rgb]{0.25,0.50,0.50}{##1}}}
\expandafter\def\csname PY@tok@cs\endcsname{\let\PY@it=\textit\def\PY@tc##1{\textcolor[rgb]{0.25,0.50,0.50}{##1}}}

\def\PYZbs{\char`\\}
\def\PYZus{\char`\_}
\def\PYZob{\char`\{}
\def\PYZcb{\char`\}}
\def\PYZca{\char`\^}
\def\PYZam{\char`\&}
\def\PYZlt{\char`\<}
\def\PYZgt{\char`\>}
\def\PYZsh{\char`\#}
\def\PYZpc{\char`\%}
\def\PYZdl{\char`\$}
\def\PYZhy{\char`\-}
\def\PYZsq{\char`\'}
\def\PYZdq{\char`\"}
\def\PYZti{\char`\~}
% for compatibility with earlier versions
\def\PYZat{@}
\def\PYZlb{[}
\def\PYZrb{]}
\makeatother


    % Exact colors from NB
    \definecolor{incolor}{rgb}{0.0, 0.0, 0.5}
    \definecolor{outcolor}{rgb}{0.545, 0.0, 0.0}



    
    % Prevent overflowing lines due to hard-to-break entities
    \sloppy 
    % Setup hyperref package
    \hypersetup{
      breaklinks=true,  % so long urls are correctly broken across lines
      colorlinks=true,
      urlcolor=urlcolor,
      linkcolor=linkcolor,
      citecolor=citecolor,
      }
    % Slightly bigger margins than the latex defaults
    
    \geometry{verbose,tmargin=1in,bmargin=1in,lmargin=1in,rmargin=1in}
    
    

    \begin{document}
    
    
    \maketitle
    
    

    
    \begin{figure}
\centering
\includegraphics{attachment:image.png}
\caption{image.png}
\end{figure}

    \section{Objetivos}\label{objetivos}

\begin{itemize}
\tightlist
\item
  Identificar los empleados con alta probabilidad de renuncia (para
  esto, usaremos el histórico de datos para encontrar patrones)
\item
  Buscar insights en los datos relacionados al personal
\item
  Encontrar un modelo óptimo para identificar a los empleados
\end{itemize}

    \begin{Verbatim}[commandchars=\\\{\}]
{\color{incolor}In [{\color{incolor}1}]:} \PY{c+c1}{\PYZsh{} importar librerias }
        \PY{k+kn}{import} \PY{n+nn}{pandas} \PY{k}{as} \PY{n+nn}{pd}
        \PY{k+kn}{import} \PY{n+nn}{numpy} \PY{k}{as} \PY{n+nn}{np}
        \PY{k+kn}{import} \PY{n+nn}{matplotlib}\PY{n+nn}{.}\PY{n+nn}{pyplot} \PY{k}{as} \PY{n+nn}{plt}
        \PY{k+kn}{import} \PY{n+nn}{seaborn} \PY{k}{as} \PY{n+nn}{sns}
        \PY{n}{sns}\PY{o}{.}\PY{n}{set}\PY{p}{(}\PY{n}{style}\PY{o}{=}\PY{l+s+s1}{\PYZsq{}}\PY{l+s+s1}{whitegrid}\PY{l+s+s1}{\PYZsq{}}\PY{p}{)}
        \PY{o}{\PYZpc{}}\PY{k}{matplotlib} inline
\end{Verbatim}


    \begin{Verbatim}[commandchars=\\\{\}]
{\color{incolor}In [{\color{incolor}2}]:} \PY{c+c1}{\PYZsh{}leer el archivo de datos a analizar}
        \PY{n}{df\PYZus{}hr} \PY{o}{=} \PY{n}{pd}\PY{o}{.}\PY{n}{read\PYZus{}csv}\PY{p}{(}\PY{l+s+s1}{\PYZsq{}}\PY{l+s+s1}{HR\PYZus{}comma\PYZus{}sep.csv}\PY{l+s+s1}{\PYZsq{}}\PY{p}{)}
        
        \PY{c+c1}{\PYZsh{}renombrar la columna referida a departamento, cambiar \PYZsq{}sales\PYZsq{} por \PYZsq{}departament\PYZsq{}}
        \PY{n}{df\PYZus{}hr}\PY{o}{.}\PY{n}{rename}\PY{p}{(}\PY{n}{columns}\PY{o}{=}\PY{p}{\PYZob{}}\PY{l+s+s1}{\PYZsq{}}\PY{l+s+s1}{sales}\PY{l+s+s1}{\PYZsq{}}\PY{p}{:} \PY{l+s+s1}{\PYZsq{}}\PY{l+s+s1}{department}\PY{l+s+s1}{\PYZsq{}}\PY{p}{\PYZcb{}}\PY{p}{,} \PY{n}{inplace}\PY{o}{=}\PY{k+kc}{True}\PY{p}{)}
        
        \PY{c+c1}{\PYZsh{} chequear summary del dataframe}
        \PY{n}{df\PYZus{}hr}\PY{o}{.}\PY{n}{info}\PY{p}{(}\PY{p}{)}
\end{Verbatim}


    \begin{Verbatim}[commandchars=\\\{\}]
<class 'pandas.core.frame.DataFrame'>
RangeIndex: 14999 entries, 0 to 14998
Data columns (total 10 columns):
satisfaction\_level       14999 non-null float64
last\_evaluation          14999 non-null float64
number\_project           14999 non-null int64
average\_montly\_hours     14999 non-null int64
time\_spend\_company       14999 non-null int64
Work\_accident            14999 non-null int64
left                     14999 non-null int64
promotion\_last\_5years    14999 non-null int64
department               14999 non-null object
salary                   14999 non-null object
dtypes: float64(2), int64(6), object(2)
memory usage: 1.1+ MB

    \end{Verbatim}

    Bien, el archivo está limpio, sin nulos, y listo para analizar

    \begin{Verbatim}[commandchars=\\\{\}]
{\color{incolor}In [{\color{incolor}3}]:} \PY{c+c1}{\PYZsh{}veamos las primeras filas del archivo para ver datos de ejemplo}
        \PY{n}{df\PYZus{}hr}\PY{o}{.}\PY{n}{head}\PY{p}{(}\PY{p}{)}
\end{Verbatim}


\begin{Verbatim}[commandchars=\\\{\}]
{\color{outcolor}Out[{\color{outcolor}3}]:}    satisfaction\_level  last\_evaluation  number\_project  average\_montly\_hours  \textbackslash{}
        0                0.38             0.53               2                   157   
        1                0.80             0.86               5                   262   
        2                0.11             0.88               7                   272   
        3                0.72             0.87               5                   223   
        4                0.37             0.52               2                   159   
        
           time\_spend\_company  Work\_accident  left  promotion\_last\_5years department  \textbackslash{}
        0                   3              0     1                      0      sales   
        1                   6              0     1                      0      sales   
        2                   4              0     1                      0      sales   
        3                   5              0     1                      0      sales   
        4                   3              0     1                      0      sales   
        
           salary  
        0     low  
        1  medium  
        2  medium  
        3     low  
        4     low  
\end{Verbatim}
            
    Haremos análisis en función al campo left('se fue de la empresa', si es
igual 1 significa que se fue, si es 0, la persona permanece en la
empresa)

    \begin{Verbatim}[commandchars=\\\{\}]
{\color{incolor}In [{\color{incolor}4}]:} \PY{c+c1}{\PYZsh{}veamos datos estadísticos descriptivos en forma de grilla}
        \PY{n}{df\PYZus{}hr}\PY{o}{.}\PY{n}{describe}\PY{p}{(}\PY{p}{)}
\end{Verbatim}


\begin{Verbatim}[commandchars=\\\{\}]
{\color{outcolor}Out[{\color{outcolor}4}]:}        satisfaction\_level  last\_evaluation  number\_project  \textbackslash{}
        count        14999.000000     14999.000000    14999.000000   
        mean             0.612834         0.716102        3.803054   
        std              0.248631         0.171169        1.232592   
        min              0.090000         0.360000        2.000000   
        25\%              0.440000         0.560000        3.000000   
        50\%              0.640000         0.720000        4.000000   
        75\%              0.820000         0.870000        5.000000   
        max              1.000000         1.000000        7.000000   
        
               average\_montly\_hours  time\_spend\_company  Work\_accident          left  \textbackslash{}
        count          14999.000000        14999.000000   14999.000000  14999.000000   
        mean             201.050337            3.498233       0.144610      0.238083   
        std               49.943099            1.460136       0.351719      0.425924   
        min               96.000000            2.000000       0.000000      0.000000   
        25\%              156.000000            3.000000       0.000000      0.000000   
        50\%              200.000000            3.000000       0.000000      0.000000   
        75\%              245.000000            4.000000       0.000000      0.000000   
        max              310.000000           10.000000       1.000000      1.000000   
        
               promotion\_last\_5years  
        count           14999.000000  
        mean                0.021268  
        std                 0.144281  
        min                 0.000000  
        25\%                 0.000000  
        50\%                 0.000000  
        75\%                 0.000000  
        max                 1.000000  
\end{Verbatim}
            
    A priori, vemos algunos datos interantes en satisfacion\_level,
last\_evaluation ypromotion\_last\_5years. Luego veremos sus curvas de
distribución para analizarlo en profundidad.

    \begin{Verbatim}[commandchars=\\\{\}]
{\color{incolor}In [{\color{incolor}5}]:} \PY{c+c1}{\PYZsh{} Veamos si existen relaciones, asociaciones (correlaciones lineales de Pearson) entre variables.}
        
        \PY{n}{plt}\PY{o}{.}\PY{n}{figure}\PY{p}{(}\PY{n}{figsize}\PY{o}{=}\PY{p}{(}\PY{l+m+mi}{10}\PY{p}{,}\PY{l+m+mi}{10}\PY{p}{)}\PY{p}{)}
        \PY{n}{sns}\PY{o}{.}\PY{n}{heatmap}\PY{p}{(}\PY{n}{data}\PY{o}{=}\PY{n}{df\PYZus{}hr}\PY{o}{.}\PY{n}{corr}\PY{p}{(}\PY{p}{)}\PY{p}{,} \PY{n}{annot}\PY{o}{=}\PY{k+kc}{True}\PY{p}{,}\PY{n}{cmap}\PY{o}{=}\PY{l+s+s2}{\PYZdq{}}\PY{l+s+s2}{YlGnBu}\PY{l+s+s2}{\PYZdq{}}\PY{p}{)}
\end{Verbatim}


\begin{Verbatim}[commandchars=\\\{\}]
{\color{outcolor}Out[{\color{outcolor}5}]:} <matplotlib.axes.\_subplots.AxesSubplot at 0x25532c49ba8>
\end{Verbatim}
            
    \begin{center}
    \adjustimage{max size={0.9\linewidth}{0.9\paperheight}}{Demo - Attrition_files/Demo - Attrition_9_1.png}
    \end{center}
    { \hspace*{\fill} \\}
    
    Ok, vemos que el nivel de satisfacción influye en la decisión de irse o
no de la empresa. También vemos otras correlaciones para analizar número
de proyectos (number\_project) y horas mensuales promedio
(average\_montly \_hours)

    \begin{Verbatim}[commandchars=\\\{\}]
{\color{incolor}In [{\color{incolor}6}]:} \PY{c+c1}{\PYZsh{}Veamos en un gráfico como se representan las distribuciones del nivel de satisfacción y el flag \PYZdq{}se fue vs no se fue\PYZdq{}}
        
        \PY{c+c1}{\PYZsh{}antes del gráfico, tomemos en dataframes aparte, la info de las personas que se fueron o no}
        \PY{n}{df\PYZus{}hr\PYZus{}left} \PY{o}{=} \PY{n}{df\PYZus{}hr}\PY{p}{[}\PY{n}{df\PYZus{}hr}\PY{p}{[}\PY{l+s+s1}{\PYZsq{}}\PY{l+s+s1}{left}\PY{l+s+s1}{\PYZsq{}}\PY{p}{]}\PY{o}{==}\PY{l+m+mi}{1}\PY{p}{]}
        \PY{n}{df\PYZus{}hr\PYZus{}not\PYZus{}left} \PY{o}{=} \PY{n}{df\PYZus{}hr}\PY{p}{[}\PY{n}{df\PYZus{}hr}\PY{p}{[}\PY{l+s+s1}{\PYZsq{}}\PY{l+s+s1}{left}\PY{l+s+s1}{\PYZsq{}}\PY{p}{]}\PY{o}{==}\PY{l+m+mi}{0}\PY{p}{]}
\end{Verbatim}


    \begin{Verbatim}[commandchars=\\\{\}]
{\color{incolor}In [{\color{incolor}7}]:} \PY{c+c1}{\PYZsh{} ahora si, veamos los gráficos de las distribuciones}
        \PY{n}{fig}\PY{p}{,} \PY{p}{(}\PY{n}{ax1}\PY{p}{,} \PY{n}{ax2}\PY{p}{)} \PY{o}{=} \PY{n}{plt}\PY{o}{.}\PY{n}{subplots}\PY{p}{(}\PY{n}{nrows}\PY{o}{=}\PY{l+m+mi}{2}\PY{p}{,} \PY{n}{figsize}\PY{o}{=}\PY{p}{(}\PY{l+m+mi}{13}\PY{p}{,}\PY{l+m+mi}{9}\PY{p}{)}\PY{p}{)}
        \PY{n}{sns}\PY{o}{.}\PY{n}{distplot}\PY{p}{(}\PY{n}{df\PYZus{}hr}\PY{p}{[}\PY{l+s+s1}{\PYZsq{}}\PY{l+s+s1}{satisfaction\PYZus{}level}\PY{l+s+s1}{\PYZsq{}}\PY{p}{]}\PY{p}{,} \PY{n}{bins}\PY{o}{=}\PY{l+m+mi}{75}\PY{p}{,} \PY{n}{ax}\PY{o}{=}\PY{n}{ax1}\PY{p}{)}
        \PY{n}{ax1}\PY{o}{.}\PY{n}{set\PYZus{}title}\PY{p}{(}\PY{l+s+s1}{\PYZsq{}}\PY{l+s+s1}{Distribución del nivel de satisfacción}\PY{l+s+s1}{\PYZsq{}}\PY{p}{)}
        
        \PY{n}{sns}\PY{o}{.}\PY{n}{kdeplot}\PY{p}{(}\PY{n}{df\PYZus{}hr\PYZus{}left}\PY{p}{[}\PY{l+s+s1}{\PYZsq{}}\PY{l+s+s1}{satisfaction\PYZus{}level}\PY{l+s+s1}{\PYZsq{}}\PY{p}{]}\PY{p}{,} \PY{n}{color}\PY{o}{=}\PY{l+s+s1}{\PYZsq{}}\PY{l+s+s1}{red}\PY{l+s+s1}{\PYZsq{}}\PY{p}{,} \PY{n}{ax}\PY{o}{=}\PY{n}{ax2}\PY{p}{,} \PY{n}{shade}\PY{o}{=}\PY{k+kc}{True}\PY{p}{)}
        \PY{n}{sns}\PY{o}{.}\PY{n}{kdeplot}\PY{p}{(}\PY{n}{df\PYZus{}hr\PYZus{}not\PYZus{}left}\PY{p}{[}\PY{l+s+s1}{\PYZsq{}}\PY{l+s+s1}{satisfaction\PYZus{}level}\PY{l+s+s1}{\PYZsq{}}\PY{p}{]}\PY{p}{,} \PY{n}{color}\PY{o}{=}\PY{l+s+s1}{\PYZsq{}}\PY{l+s+s1}{green}\PY{l+s+s1}{\PYZsq{}}\PY{p}{,} \PY{n}{ax}\PY{o}{=}\PY{n}{ax2}\PY{p}{,} \PY{n}{shade}\PY{o}{=}\PY{k+kc}{True}\PY{p}{)}
        \PY{n}{ax2}\PY{o}{.}\PY{n}{set\PYZus{}title}\PY{p}{(}\PY{l+s+s1}{\PYZsq{}}\PY{l+s+s1}{Nivel de satisfacción: Se fue vs No se fue}\PY{l+s+s1}{\PYZsq{}}\PY{p}{)}
        \PY{n}{ax2}\PY{o}{.}\PY{n}{legend}\PY{p}{(}\PY{p}{[}\PY{l+s+s1}{\PYZsq{}}\PY{l+s+s1}{se fue}\PY{l+s+s1}{\PYZsq{}}\PY{p}{,} \PY{l+s+s1}{\PYZsq{}}\PY{l+s+s1}{no se fue}\PY{l+s+s1}{\PYZsq{}}\PY{p}{]}\PY{p}{)}
\end{Verbatim}


\begin{Verbatim}[commandchars=\\\{\}]
{\color{outcolor}Out[{\color{outcolor}7}]:} <matplotlib.legend.Legend at 0x25532cd4518>
\end{Verbatim}
            
    \begin{center}
    \adjustimage{max size={0.9\linewidth}{0.9\paperheight}}{Demo - Attrition_files/Demo - Attrition_12_1.png}
    \end{center}
    { \hspace*{\fill} \\}
    
    ok, se ve que las clases presentan una separación - cuando el nivel de
satisfacción es \textgreater{} 0.5, vemos que menos gente se va - por
otro lado, el pico máximo de las personas que se fueron, tuvieron un
nivel de satisfacción alrededor de 0.4 de satisfacción.

Datos interesantes a tener en cuenta para seguir el análisis.

    \begin{Verbatim}[commandchars=\\\{\}]
{\color{incolor}In [{\color{incolor}8}]:} \PY{c+c1}{\PYZsh{}veamos las variables a lo largo de todos los departamentos}
        \PY{n}{department}\PY{o}{=}\PY{n}{df\PYZus{}hr}\PY{o}{.}\PY{n}{groupby}\PY{p}{(}\PY{l+s+s1}{\PYZsq{}}\PY{l+s+s1}{department}\PY{l+s+s1}{\PYZsq{}}\PY{p}{)}\PY{o}{.}\PY{n}{mean}\PY{p}{(}\PY{p}{)}
        \PY{n}{department}
\end{Verbatim}


\begin{Verbatim}[commandchars=\\\{\}]
{\color{outcolor}Out[{\color{outcolor}8}]:}              satisfaction\_level  last\_evaluation  number\_project  \textbackslash{}
        department                                                         
        IT                     0.618142         0.716830        3.816626   
        RandD                  0.619822         0.712122        3.853875   
        accounting             0.582151         0.717718        3.825293   
        hr                     0.598809         0.708850        3.654939   
        management             0.621349         0.724000        3.860317   
        marketing              0.618601         0.715886        3.687646   
        product\_mng            0.619634         0.714756        3.807095   
        sales                  0.614447         0.709717        3.776329   
        support                0.618300         0.723109        3.803948   
        technical              0.607897         0.721099        3.877941   
        
                     average\_montly\_hours  time\_spend\_company  Work\_accident  \textbackslash{}
        department                                                             
        IT                     202.215974            3.468623       0.133659   
        RandD                  200.800508            3.367217       0.170267   
        accounting             201.162973            3.522816       0.125163   
        hr                     198.684709            3.355886       0.120433   
        management             201.249206            4.303175       0.163492   
        marketing              199.385781            3.569930       0.160839   
        product\_mng            199.965632            3.475610       0.146341   
        sales                  200.911353            3.534058       0.141787   
        support                200.758188            3.393001       0.154778   
        technical              202.497426            3.411397       0.140074   
        
                         left  promotion\_last\_5years  
        department                                    
        IT           0.222494               0.002445  
        RandD        0.153748               0.034307  
        accounting   0.265971               0.018253  
        hr           0.290934               0.020298  
        management   0.144444               0.109524  
        marketing    0.236597               0.050117  
        product\_mng  0.219512               0.000000  
        sales        0.244928               0.024155  
        support      0.248991               0.008973  
        technical    0.256250               0.010294  
\end{Verbatim}
            
    \subsection{Analicemos en particular las personas que se
fueron}\label{analicemos-en-particular-las-personas-que-se-fueron}

    \begin{Verbatim}[commandchars=\\\{\}]
{\color{incolor}In [{\color{incolor}9}]:} \PY{n}{df\PYZus{}hr\PYZus{}left}\PY{o}{.}\PY{n}{info}\PY{p}{(}\PY{p}{)}
\end{Verbatim}


    \begin{Verbatim}[commandchars=\\\{\}]
<class 'pandas.core.frame.DataFrame'>
Int64Index: 3571 entries, 0 to 14998
Data columns (total 10 columns):
satisfaction\_level       3571 non-null float64
last\_evaluation          3571 non-null float64
number\_project           3571 non-null int64
average\_montly\_hours     3571 non-null int64
time\_spend\_company       3571 non-null int64
Work\_accident            3571 non-null int64
left                     3571 non-null int64
promotion\_last\_5years    3571 non-null int64
department               3571 non-null object
salary                   3571 non-null object
dtypes: float64(2), int64(6), object(2)
memory usage: 306.9+ KB

    \end{Verbatim}

    \begin{Verbatim}[commandchars=\\\{\}]
{\color{incolor}In [{\color{incolor}10}]:} \PY{c+c1}{\PYZsh{}veamos la matriz de correlación solo de las personas que se fueron}
         
         \PY{n}{plt}\PY{o}{.}\PY{n}{figure}\PY{p}{(}\PY{n}{figsize}\PY{o}{=}\PY{p}{(}\PY{l+m+mi}{10}\PY{p}{,}\PY{l+m+mi}{10}\PY{p}{)}\PY{p}{)}
         \PY{n}{sns}\PY{o}{.}\PY{n}{heatmap}\PY{p}{(}\PY{n}{data}\PY{o}{=}\PY{n}{df\PYZus{}hr\PYZus{}left}\PY{p}{[}\PY{n}{df\PYZus{}hr\PYZus{}left}\PY{o}{.}\PY{n}{columns}\PY{o}{.}\PY{n}{difference}\PY{p}{(}\PY{p}{[}\PY{l+s+s1}{\PYZsq{}}\PY{l+s+s1}{left}\PY{l+s+s1}{\PYZsq{}}\PY{p}{]}\PY{p}{)}\PY{p}{]}\PY{o}{.}\PY{n}{corr}\PY{p}{(}\PY{l+s+s1}{\PYZsq{}}\PY{l+s+s1}{pearson}\PY{l+s+s1}{\PYZsq{}}\PY{p}{)}\PY{p}{,} \PY{n}{annot}\PY{o}{=}\PY{k+kc}{True}\PY{p}{,}\PY{n}{cmap}\PY{o}{=}\PY{l+s+s2}{\PYZdq{}}\PY{l+s+s2}{YlGnBu}\PY{l+s+s2}{\PYZdq{}}\PY{p}{)}
\end{Verbatim}


\begin{Verbatim}[commandchars=\\\{\}]
{\color{outcolor}Out[{\color{outcolor}10}]:} <matplotlib.axes.\_subplots.AxesSubplot at 0x2553301ca20>
\end{Verbatim}
            
    \begin{center}
    \adjustimage{max size={0.9\linewidth}{0.9\paperheight}}{Demo - Attrition_files/Demo - Attrition_17_1.png}
    \end{center}
    { \hspace*{\fill} \\}
    
    ok, vemos asociaciones interesantes: promedio\_horas\_mensuales y ultima
evaluacion/numero\_de\_proyectos tienen una relacion muy fuerte

    \begin{Verbatim}[commandchars=\\\{\}]
{\color{incolor}In [{\color{incolor}11}]:} \PY{c+c1}{\PYZsh{}continuemos analizando la gente que se fue y sus características.}
         
         \PY{c+c1}{\PYZsh{}Comencemos viendo la distribución de última evaluacion.}
         \PY{n}{ax} \PY{o}{=} \PY{n}{sns}\PY{o}{.}\PY{n}{kdeplot}\PY{p}{(}\PY{n}{df\PYZus{}hr\PYZus{}left}\PY{o}{.}\PY{n}{last\PYZus{}evaluation}\PY{p}{,} \PY{n}{shade}\PY{o}{=}\PY{k+kc}{True}\PY{p}{,} \PY{n}{color}\PY{o}{=}\PY{l+s+s2}{\PYZdq{}}\PY{l+s+s2}{r}\PY{l+s+s2}{\PYZdq{}} \PY{p}{)}
\end{Verbatim}


    \begin{center}
    \adjustimage{max size={0.9\linewidth}{0.9\paperheight}}{Demo - Attrition_files/Demo - Attrition_19_0.png}
    \end{center}
    { \hspace*{\fill} \\}
    
    Aqui vemos dos picos interesantes. el mas importante es el de la derecha
donde encontramos personas con una alta nota en la última evaluación y
se fue de la empresa. Del lado izquierdo vemos un pico con un promedio
de nota mas bajo. Aqui podemos separar en buenos empleados (nota
\textgreater{} .7) y no tan buenos (nota \textless{} .7)

    \subsubsection{El salario y la cantidad de años en la empresa son
predictores de
renuncia?}\label{el-salario-y-la-cantidad-de-auxf1os-en-la-empresa-son-predictores-de-renuncia}

    \begin{Verbatim}[commandchars=\\\{\}]
{\color{incolor}In [{\color{incolor}12}]:} \PY{c+c1}{\PYZsh{}veamos el salario usando gráfico de barras}
         
         \PY{n}{sns}\PY{o}{.}\PY{n}{countplot}\PY{p}{(}\PY{n}{x}\PY{o}{=}\PY{l+s+s1}{\PYZsq{}}\PY{l+s+s1}{salary}\PY{l+s+s1}{\PYZsq{}}\PY{p}{,} \PY{n}{data}\PY{o}{=} \PY{n}{df\PYZus{}hr} \PY{p}{,} \PY{n}{hue}\PY{o}{=}\PY{l+s+s1}{\PYZsq{}}\PY{l+s+s1}{left}\PY{l+s+s1}{\PYZsq{}}\PY{p}{)}
\end{Verbatim}


\begin{Verbatim}[commandchars=\\\{\}]
{\color{outcolor}Out[{\color{outcolor}12}]:} <matplotlib.axes.\_subplots.AxesSubplot at 0x2553390f208>
\end{Verbatim}
            
    \begin{center}
    \adjustimage{max size={0.9\linewidth}{0.9\paperheight}}{Demo - Attrition_files/Demo - Attrition_22_1.png}
    \end{center}
    { \hspace*{\fill} \\}
    
    ok, aqui vemos que significativamente los empleados con sueldos bajo o
medio renuncian mas en proporción a los empleados que mas cobran. Tiene
sentido.

    \begin{Verbatim}[commandchars=\\\{\}]
{\color{incolor}In [{\color{incolor}13}]:} \PY{c+c1}{\PYZsh{}veamos la cantidad de año en la empresa usando diagramas de cajas}
         \PY{n}{sns}\PY{o}{.}\PY{n}{countplot}\PY{p}{(}\PY{n}{x}\PY{o}{=}\PY{l+s+s1}{\PYZsq{}}\PY{l+s+s1}{time\PYZus{}spend\PYZus{}company}\PY{l+s+s1}{\PYZsq{}}\PY{p}{,} \PY{n}{data}\PY{o}{=} \PY{n}{df\PYZus{}hr} \PY{p}{,} \PY{n}{hue}\PY{o}{=}\PY{l+s+s1}{\PYZsq{}}\PY{l+s+s1}{left}\PY{l+s+s1}{\PYZsq{}}\PY{p}{)}
\end{Verbatim}


\begin{Verbatim}[commandchars=\\\{\}]
{\color{outcolor}Out[{\color{outcolor}13}]:} <matplotlib.axes.\_subplots.AxesSubplot at 0x25533735cf8>
\end{Verbatim}
            
    \begin{center}
    \adjustimage{max size={0.9\linewidth}{0.9\paperheight}}{Demo - Attrition_files/Demo - Attrition_24_1.png}
    \end{center}
    { \hspace*{\fill} \\}
    
    En verde, las cantidad d elos que se fueron. Vemos que entre 3 y 6 años,
la proporcion de personas que dejaron la empresa es alta.

    \subsubsection{Veamos como se relacionan el nivel de satisfacción y la
última
evaluación}\label{veamos-como-se-relacionan-el-nivel-de-satisfacciuxf3n-y-la-uxfaltima-evaluaciuxf3n}

    \begin{Verbatim}[commandchars=\\\{\}]
{\color{incolor}In [{\color{incolor}14}]:} \PY{n}{sns}\PY{o}{.}\PY{n}{lmplot}\PY{p}{(}\PY{n}{x}\PY{o}{=}\PY{l+s+s2}{\PYZdq{}}\PY{l+s+s2}{last\PYZus{}evaluation}\PY{l+s+s2}{\PYZdq{}}\PY{p}{,} \PY{n}{y}\PY{o}{=}\PY{l+s+s2}{\PYZdq{}}\PY{l+s+s2}{satisfaction\PYZus{}level}\PY{l+s+s2}{\PYZdq{}}\PY{p}{,} \PY{n}{data}\PY{o}{=}\PY{n}{df\PYZus{}hr}\PY{p}{,} \PY{n}{hue}\PY{o}{=}\PY{l+s+s1}{\PYZsq{}}\PY{l+s+s1}{left}\PY{l+s+s1}{\PYZsq{}}\PY{p}{,} \PY{n}{fit\PYZus{}reg}\PY{o}{=}\PY{k+kc}{False}\PY{p}{)}
\end{Verbatim}


\begin{Verbatim}[commandchars=\\\{\}]
{\color{outcolor}Out[{\color{outcolor}14}]:} <seaborn.axisgrid.FacetGrid at 0x25533760e10>
\end{Verbatim}
            
    \begin{center}
    \adjustimage{max size={0.9\linewidth}{0.9\paperheight}}{Demo - Attrition_files/Demo - Attrition_27_1.png}
    \end{center}
    { \hspace*{\fill} \\}
    
    Interesante, vemos que existen 3 grupos que se fueron (en verde) con
diferentes características en función al nivel de satisfacción - Nivel
de satisfacción promedio .4 y última evaluación promedio .5 -\/-a priori
no serian los relevantes a revisar - Nivel de satisfacción promedio
\textless{} .2 y alta nota de evaluación \textgreater{} .7 -\/- Riesgo
alto. Pueden tratarse de grandes empleados poco o nada satisfechos y con
alto grado de conocimiento. Alerta. - Nivel de satisfacción alto
\textgreater{} .7 y alta nota de evaluación \textgreater{} .7 -\/-
buenos empleados y satisfechos. Quizas se fueron por mejores ofertas?

    \subsection{Bien, ahora buscaremos el mejor modelo de Machine Learning
(ML) para identificar a las probables renuncias a
futuro}\label{bien-ahora-buscaremos-el-mejor-modelo-de-machine-learning-ml-para-identificar-a-las-probables-renuncias-a-futuro}

    \subsubsection{Primero usaremos el modelo de Regresión logística para
clasificar las personas que se pueden
ir}\label{primero-usaremos-el-modelo-de-regresiuxf3n-loguxedstica-para-clasificar-las-personas-que-se-pueden-ir}

    \begin{Verbatim}[commandchars=\\\{\}]
{\color{incolor}In [{\color{incolor}15}]:} \PY{c+c1}{\PYZsh{}antes de crear el modelo, pasamos a numérico la variable categórica Salario}
         \PY{n}{map\PYZus{}salary} \PY{o}{=} \PY{p}{\PYZob{}}\PY{l+s+s1}{\PYZsq{}}\PY{l+s+s1}{low}\PY{l+s+s1}{\PYZsq{}}\PY{p}{:} \PY{l+m+mi}{0}\PY{p}{,} \PY{l+s+s1}{\PYZsq{}}\PY{l+s+s1}{medium}\PY{l+s+s1}{\PYZsq{}}\PY{p}{:} \PY{l+m+mi}{1}\PY{p}{,} \PY{l+s+s1}{\PYZsq{}}\PY{l+s+s1}{high}\PY{l+s+s1}{\PYZsq{}}\PY{p}{:} \PY{l+m+mi}{2}\PY{p}{\PYZcb{}} 
         \PY{n}{df\PYZus{}hr}\PY{o}{.}\PY{n}{replace}\PY{p}{(}\PY{p}{\PYZob{}}\PY{l+s+s1}{\PYZsq{}}\PY{l+s+s1}{salary}\PY{l+s+s1}{\PYZsq{}}\PY{p}{:} \PY{n}{map\PYZus{}salary}\PY{p}{\PYZcb{}}\PY{p}{,} \PY{n}{inplace}\PY{o}{=}\PY{k+kc}{True}\PY{p}{)}
\end{Verbatim}


    \begin{Verbatim}[commandchars=\\\{\}]
{\color{incolor}In [{\color{incolor}16}]:} \PY{o}{\PYZpc{}\PYZpc{}}\PY{k}{time}
         from sklearn import preprocessing
         from sklearn.linear\PYZus{}model import LogisticRegression
         from sklearn.model\PYZus{}selection import train\PYZus{}test\PYZus{}split, KFold, cross\PYZus{}val\PYZus{}score, GridSearchCV
         
         X\PYZus{}lr = df\PYZus{}hr.drop([\PYZsq{}left\PYZsq{}, \PYZsq{}department\PYZsq{}], axis=1)
         y\PYZus{}lr = df\PYZus{}hr[\PYZsq{}left\PYZsq{}]
         X\PYZus{}lr = preprocessing.scale(X\PYZus{}lr)
         
         \PYZsh{}usaremos como técnica anti\PYZhy{}overfitting el método split\PYZhy{}test
         X\PYZus{}train, X\PYZus{}test, y\PYZus{}train, y\PYZus{}test = train\PYZus{}test\PYZus{}split(X\PYZus{}lr,y\PYZus{}lr,test\PYZus{}size = 0.3, random\PYZus{}state = 0)
         
         model\PYZus{}lr = LogisticRegression()
         param\PYZus{}lr = \PYZob{}\PYZsq{}C\PYZsq{} : np.linspace(1, 10, 20)\PYZcb{}
         
         clf\PYZus{}lr = GridSearchCV(model\PYZus{}lr, param\PYZus{}lr, cv=5, return\PYZus{}train\PYZus{}score=False)
         
         clf\PYZus{}lr.fit(X\PYZus{}lr, y\PYZus{}lr)
         
         print(\PYZsq{}La precisión promedio para Regression Logística es es: \PYZob{}\PYZcb{}\PYZsq{}.format(round(clf\PYZus{}lr.best\PYZus{}score\PYZus{}, 5)))
\end{Verbatim}


    \begin{Verbatim}[commandchars=\\\{\}]
La precisión promedio para Regression Logística es es: 0.77192
Wall time: 1.58 s

    \end{Verbatim}

    A Priori 77\% es una métrica de accuracy regular, pero veamos las
valores de performance

    \begin{Verbatim}[commandchars=\\\{\}]
{\color{incolor}In [{\color{incolor}17}]:} \PY{c+c1}{\PYZsh{}creamos la serie con la nueva clase}
         \PY{n}{y\PYZus{}pred} \PY{o}{=} \PY{n}{clf\PYZus{}lr}\PY{o}{.}\PY{n}{predict}\PY{p}{(}\PY{n}{X\PYZus{}test}\PY{p}{)}
\end{Verbatim}


    \begin{Verbatim}[commandchars=\\\{\}]
{\color{incolor}In [{\color{incolor}18}]:} \PY{c+c1}{\PYZsh{}veamos la matriz de confusión, y las metricas}
         
         \PY{k+kn}{from} \PY{n+nn}{sklearn} \PY{k}{import} \PY{n}{metrics}
         \PY{n+nb}{print}\PY{p}{(}\PY{n}{metrics}\PY{o}{.}\PY{n}{confusion\PYZus{}matrix}\PY{p}{(}\PY{n}{y\PYZus{}test}\PY{p}{,}\PY{n}{y\PYZus{}pred}\PY{p}{)}\PY{p}{)}
         \PY{n+nb}{print}\PY{p}{(}\PY{n}{metrics}\PY{o}{.}\PY{n}{recall\PYZus{}score}\PY{p}{(}\PY{n}{y\PYZus{}test}\PY{p}{,}\PY{n}{y\PYZus{}pred}\PY{p}{)}\PY{p}{)}
         \PY{n+nb}{print}\PY{p}{(}\PY{n}{metrics}\PY{o}{.}\PY{n}{precision\PYZus{}score}\PY{p}{(}\PY{n}{y\PYZus{}test}\PY{p}{,}\PY{n}{y\PYZus{}pred}\PY{p}{)}\PY{p}{)}
\end{Verbatim}


    \begin{Verbatim}[commandchars=\\\{\}]
[[3200  262]
 [ 662  376]]
0.362235067437
0.58934169279

    \end{Verbatim}

    Mmm... si bien cerca del 80\% el modelo acierta el resultado, para el
caso nuestro, donde nos interesa si acierta en los casos que la persona
se va de la empresa, el porcentaje de sensitividad (recall) es muy bajo,
solo un 36\% de las veces el modelo acierta que si realmente se fue,
nosotros acertamos. Si bien, podriamos bajar el umbral para lograr mas
precisión, veamos otro modelo de ML, porque la performance de este
modelo es mala. El modelo no es robusto, y probaremos con otra técnica!

    \subsubsection{Ahora veamos el potente algoritmo de clasificación SVM
(Support Vector
Machine)}\label{ahora-veamos-el-potente-algoritmo-de-clasificaciuxf3n-svm-support-vector-machine}

    \begin{Verbatim}[commandchars=\\\{\}]
{\color{incolor}In [{\color{incolor}19}]:} \PY{c+c1}{\PYZsh{} Debajo usaremos cross\PYZus{}validation para evitar overfitting}
\end{Verbatim}


    \begin{Verbatim}[commandchars=\\\{\}]
{\color{incolor}In [{\color{incolor}20}]:} \PY{o}{\PYZpc{}\PYZpc{}}\PY{k}{time}
         from sklearn.svm import SVC
         
         X\PYZus{}svc = df\PYZus{}hr.drop([\PYZsq{}left\PYZsq{}, \PYZsq{}department\PYZsq{}], axis=1)
         y\PYZus{}svc = df\PYZus{}hr[\PYZsq{}left\PYZsq{}]
         
         X\PYZus{}svc = preprocessing.scale(X\PYZus{}svc)
         
         \PYZsh{}ejecutamos support vector classifier con los hyperparámetros por defecto
         model\PYZus{}svc = SVC(probability=True)
         
         scores = cross\PYZus{}val\PYZus{}score(model\PYZus{}svc,X\PYZus{}svc,y\PYZus{}svc,cv=10,scoring=\PYZdq{}recall\PYZdq{})
         
         print(\PYZsq{}El accuracy, la cantidad de aciertos es de: \PYZsq{},cross\PYZus{}val\PYZus{}score(model\PYZus{}svc,X\PYZus{}svc,y\PYZus{}svc,cv=10,scoring=\PYZdq{}accuracy\PYZdq{}).mean())
         model\PYZus{}svc.fit(X\PYZus{}svc, y\PYZus{}svc)
         
         print(\PYZsq{}La media del recall \PYZhy{}sensitivity\PYZhy{} es: \PYZsq{},scores.mean()) 
\end{Verbatim}


    \begin{Verbatim}[commandchars=\\\{\}]
El accuracy, la cantidad de aciertos es de:  0.964865240473
La media del recall -sensitivity- es:  0.909265605684
Wall time: 1min 24s

    \end{Verbatim}

    Aqui vemos que casi el 91\% de las veces que la persona se fue, nosotros
predecimos lo mismo. Es un muy buen ratio para el modelo

    \begin{Verbatim}[commandchars=\\\{\}]
{\color{incolor}In [{\color{incolor}21}]:} \PY{n+nb}{print}\PY{p}{(}\PY{l+s+s1}{\PYZsq{}}\PY{l+s+s1}{La precision, la cantidad de negativos acertados es de: }\PY{l+s+s1}{\PYZsq{}}\PY{p}{,}\PY{n}{cross\PYZus{}val\PYZus{}score}\PY{p}{(}\PY{n}{model\PYZus{}svc}\PY{p}{,}\PY{n}{X\PYZus{}svc}\PY{p}{,}\PY{n}{y\PYZus{}svc}\PY{p}{,}\PY{n}{cv}\PY{o}{=}\PY{l+m+mi}{10}\PY{p}{,}\PY{n}{scoring}\PY{o}{=}\PY{l+s+s2}{\PYZdq{}}\PY{l+s+s2}{precision}\PY{l+s+s2}{\PYZdq{}}\PY{p}{)}\PY{o}{.}\PY{n}{mean}\PY{p}{(}\PY{p}{)}\PY{p}{)}
\end{Verbatim}


    \begin{Verbatim}[commandchars=\\\{\}]
La precision, la cantidad de negativos acertados es de:  0.941542737242

    \end{Verbatim}

    \begin{Verbatim}[commandchars=\\\{\}]
{\color{incolor}In [{\color{incolor}22}]:} \PY{k+kn}{from} \PY{n+nn}{sklearn}\PY{n+nn}{.}\PY{n+nn}{model\PYZus{}selection} \PY{k}{import} \PY{n}{cross\PYZus{}val\PYZus{}predict}
         \PY{k+kn}{from} \PY{n+nn}{sklearn}\PY{n+nn}{.}\PY{n+nn}{metrics} \PY{k}{import} \PY{n}{confusion\PYZus{}matrix}
         \PY{n}{y\PYZus{}pred\PYZus{}svc} \PY{o}{=} \PY{n}{cross\PYZus{}val\PYZus{}predict}\PY{p}{(}\PY{n}{model\PYZus{}svc}\PY{p}{,}\PY{n}{X\PYZus{}svc}\PY{p}{,}\PY{n}{y\PYZus{}svc}\PY{p}{,}\PY{n}{cv}\PY{o}{=}\PY{l+m+mi}{10}\PY{p}{)}
         \PY{n}{conf\PYZus{}mat\PYZus{}svc} \PY{o}{=} \PY{n}{confusion\PYZus{}matrix}\PY{p}{(}\PY{n}{y\PYZus{}svc}\PY{p}{,}\PY{n}{y\PYZus{}pred\PYZus{}svc}\PY{p}{)}
\end{Verbatim}


    \begin{Verbatim}[commandchars=\\\{\}]
{\color{incolor}In [{\color{incolor}23}]:} \PY{n}{conf\PYZus{}mat\PYZus{}svc}
\end{Verbatim}


\begin{Verbatim}[commandchars=\\\{\}]
{\color{outcolor}Out[{\color{outcolor}23}]:} array([[11225,   203],
                [  324,  3247]], dtype=int64)
\end{Verbatim}
            
    Mirando la matriz de confusión, se corroboran las métricas, y se
observan muy buenos números en relacion target-predicted.

    \begin{Verbatim}[commandchars=\\\{\}]
{\color{incolor}In [{\color{incolor}24}]:} \PY{n}{fpr}\PY{p}{,} \PY{n}{tpr}\PY{p}{,} \PY{n}{thresholds} \PY{o}{=} \PY{n}{metrics}\PY{o}{.}\PY{n}{roc\PYZus{}curve}\PY{p}{(}\PY{n}{y\PYZus{}svc}\PY{p}{,}\PY{n}{y\PYZus{}pred\PYZus{}svc}\PY{p}{)}
\end{Verbatim}


    \begin{Verbatim}[commandchars=\\\{\}]
{\color{incolor}In [{\color{incolor}25}]:} \PY{n}{plt}\PY{o}{.}\PY{n}{plot}\PY{p}{(}\PY{n}{fpr}\PY{p}{,}\PY{n}{tpr}\PY{p}{)}
         \PY{n}{plt}\PY{o}{.}\PY{n}{xlim}\PY{p}{(}\PY{p}{[}\PY{l+m+mf}{0.0}\PY{p}{,}\PY{l+m+mf}{1.0}\PY{p}{]}\PY{p}{)}
         \PY{n}{plt}\PY{o}{.}\PY{n}{ylim}\PY{p}{(}\PY{p}{[}\PY{l+m+mf}{0.0}\PY{p}{,}\PY{l+m+mf}{1.0}\PY{p}{]}\PY{p}{)}
         \PY{n}{plt}\PY{o}{.}\PY{n}{title}\PY{p}{(}\PY{l+s+s1}{\PYZsq{}}\PY{l+s+s1}{ROC curva}\PY{l+s+s1}{\PYZsq{}}\PY{p}{)}
         \PY{n}{plt}\PY{o}{.}\PY{n}{xlabel}\PY{p}{(}\PY{l+s+s1}{\PYZsq{}}\PY{l+s+s1}{False positive rate}\PY{l+s+s1}{\PYZsq{}}\PY{p}{)}
         \PY{n}{plt}\PY{o}{.}\PY{n}{ylabel}\PY{p}{(}\PY{l+s+s1}{\PYZsq{}}\PY{l+s+s1}{True positive rate. Sensitividad}\PY{l+s+s1}{\PYZsq{}}\PY{p}{)}
\end{Verbatim}


\begin{Verbatim}[commandchars=\\\{\}]
{\color{outcolor}Out[{\color{outcolor}25}]:} Text(0,0.5,'True positive rate. Sensitividad')
\end{Verbatim}
            
    \begin{center}
    \adjustimage{max size={0.9\linewidth}{0.9\paperheight}}{Demo - Attrition_files/Demo - Attrition_46_1.png}
    \end{center}
    { \hspace*{\fill} \\}
    
    La curva ROC tiene muy buena performance, como se ve en la imagen. Alto
grado de aciertos en true positives, y pocos false positives.

    \begin{Verbatim}[commandchars=\\\{\}]
{\color{incolor}In [{\color{incolor}26}]:} \PY{c+c1}{\PYZsh{}Abajo podemos confirmarlo con el área cubierta bajo la curva.}
         \PY{n+nb}{print}\PY{p}{(}\PY{l+s+s1}{\PYZsq{}}\PY{l+s+s1}{ROC AUC: }\PY{l+s+s1}{\PYZsq{}}\PY{p}{,}\PY{n}{cross\PYZus{}val\PYZus{}score}\PY{p}{(}\PY{n}{model\PYZus{}svc}\PY{p}{,}\PY{n}{X\PYZus{}svc}\PY{p}{,}\PY{n}{y\PYZus{}svc}\PY{p}{,}\PY{n}{cv}\PY{o}{=}\PY{l+m+mi}{10}\PY{p}{,}\PY{n}{scoring}\PY{o}{=}\PY{l+s+s2}{\PYZdq{}}\PY{l+s+s2}{roc\PYZus{}auc}\PY{l+s+s2}{\PYZdq{}}\PY{p}{)}\PY{o}{.}\PY{n}{mean}\PY{p}{(}\PY{p}{)}\PY{p}{)}
\end{Verbatim}


    \begin{Verbatim}[commandchars=\\\{\}]
ROC AUC:  0.977411390032

    \end{Verbatim}

    \begin{Verbatim}[commandchars=\\\{\}]
{\color{incolor}In [{\color{incolor}27}]:} \PY{c+c1}{\PYZsh{}veamos las probabilidades de salida}
         \PY{n}{y\PYZus{}pred\PYZus{}svc\PYZus{}proba} \PY{o}{=} \PY{n}{model\PYZus{}svc}\PY{o}{.}\PY{n}{predict\PYZus{}proba}\PY{p}{(}\PY{n}{X\PYZus{}svc}\PY{p}{)}
\end{Verbatim}


    \begin{Verbatim}[commandchars=\\\{\}]
{\color{incolor}In [{\color{incolor}28}]:} \PY{n}{predicted\PYZus{}values\PYZus{}svc} \PY{o}{=} \PY{n}{pd}\PY{o}{.}\PY{n}{Series}\PY{p}{(}\PY{n}{model\PYZus{}svc}\PY{o}{.}\PY{n}{predict}\PY{p}{(}\PY{n}{X\PYZus{}svc}\PY{p}{)}\PY{p}{,} \PY{n}{index}\PY{o}{=}\PY{n}{y\PYZus{}svc}\PY{o}{.}\PY{n}{index}\PY{p}{)}
\end{Verbatim}


    \begin{Verbatim}[commandchars=\\\{\}]
{\color{incolor}In [{\color{incolor}29}]:} \PY{n}{proba\PYZus{}values\PYZus{}svc} \PY{o}{=} \PY{n}{pd}\PY{o}{.}\PY{n}{Series}\PY{p}{(}\PY{n}{model\PYZus{}svc}\PY{o}{.}\PY{n}{predict\PYZus{}proba}\PY{p}{(}\PY{n}{X\PYZus{}svc}\PY{p}{)}\PY{p}{[}\PY{p}{:}\PY{p}{,}\PY{l+m+mi}{1}\PY{p}{]}\PY{p}{,} \PY{n}{index}\PY{o}{=}\PY{n}{y\PYZus{}svc}\PY{o}{.}\PY{n}{index}\PY{p}{)}
\end{Verbatim}


    \begin{Verbatim}[commandchars=\\\{\}]
{\color{incolor}In [{\color{incolor}30}]:} \PY{c+c1}{\PYZsh{}agrupamos en un vector los empleados que permanecen en la empresa con riesgo de irse}
         \PY{n}{empleados\PYZus{}con\PYZus{}riesgo} \PY{o}{=} \PY{p}{[}\PY{p}{]}
         \PY{k}{for} \PY{n}{index} \PY{o+ow}{in} \PY{n}{y\PYZus{}svc}\PY{o}{.}\PY{n}{index}\PY{p}{:}
             \PY{k}{if} \PY{p}{(}\PY{n}{predicted\PYZus{}values\PYZus{}svc}\PY{o}{.}\PY{n}{loc}\PY{p}{[}\PY{n}{index}\PY{p}{]} \PY{o}{==} \PY{l+m+mi}{1}\PY{p}{)} \PY{o+ow}{and} \PY{p}{(}\PY{n}{y\PYZus{}svc}\PY{o}{.}\PY{n}{loc}\PY{p}{[}\PY{n}{index}\PY{p}{]} \PY{o}{==} \PY{l+m+mi}{0}\PY{p}{)}\PY{p}{:}
                 \PY{n}{empleados\PYZus{}con\PYZus{}riesgo}\PY{o}{.}\PY{n}{append}\PY{p}{(}\PY{n}{index}\PY{p}{)}
\end{Verbatim}


    \begin{Verbatim}[commandchars=\\\{\}]
{\color{incolor}In [{\color{incolor}31}]:} \PY{n}{df\PYZus{}empleados\PYZus{}con\PYZus{}riesgo} \PY{o}{=} \PY{n}{pd}\PY{o}{.}\PY{n}{DataFrame}\PY{p}{(}\PY{n}{index}\PY{o}{=}\PY{n}{empleados\PYZus{}con\PYZus{}riesgo}\PY{p}{)}
\end{Verbatim}


    \begin{Verbatim}[commandchars=\\\{\}]
{\color{incolor}In [{\color{incolor}32}]:} \PY{n}{df\PYZus{}empleados\PYZus{}con\PYZus{}riesgo}\PY{p}{[}\PY{l+s+s1}{\PYZsq{}}\PY{l+s+s1}{prob\PYZus{}renuncia}\PY{l+s+s1}{\PYZsq{}}\PY{p}{]} \PY{o}{=} \PY{n}{proba\PYZus{}values\PYZus{}svc}
\end{Verbatim}


    \begin{Verbatim}[commandchars=\\\{\}]
{\color{incolor}In [{\color{incolor}33}]:} \PY{n+nb}{print}\PY{p}{(}\PY{l+s+s1}{\PYZsq{}}\PY{l+s+s1}{Los ids de los empleados con riesgo de irse son: }\PY{l+s+si}{\PYZob{}\PYZcb{}}\PY{l+s+s1}{ .}\PY{l+s+s1}{\PYZsq{}}\PY{o}{.}\PY{n}{format}\PY{p}{(}\PY{n}{empleados\PYZus{}con\PYZus{}riesgo}\PY{p}{)}\PY{p}{)}
\end{Verbatim}


    \begin{Verbatim}[commandchars=\\\{\}]
Los ids de los empleados con riesgo de irse son: [2011, 2094, 2206, 2214, 2410, 2415, 2421, 2475, 2498, 2550, 2570, 2703, 2723, 2801, 2842, 3038, 3056, 3079, 3106, 3114, 3160, 3326, 3345, 3518, 3710, 3780, 3890, 3921, 3989, 4076, 4240, 4392, 4455, 4519, 4534, 4621, 4673, 4774, 4783, 4830, 4837, 4843, 4879, 4892, 4932, 4982, 5072, 5104, 5149, 5151, 5180, 5270, 5356, 5431, 5436, 5565, 5762, 5840, 5847, 5944, 5949, 5994, 6012, 6055, 6056, 6076, 6086, 6189, 6228, 6263, 6290, 6319, 6358, 6448, 6457, 6466, 6472, 6497, 6644, 6726, 6728, 6811, 6826, 6854, 6896, 6928, 7004, 7006, 7077, 7135, 7223, 7241, 7244, 7251, 7313, 7323, 7355, 7384, 7430, 7443, 7560, 7727, 7732, 7745, 7762, 7782, 7805, 7818, 7918, 7983, 7989, 8010, 8022, 8128, 8157, 8193, 8257, 8270, 8301, 8335, 8394, 8416, 8451, 8489, 8522, 8553, 8591, 8682, 8741, 8891, 8895, 9059, 9171, 9175, 9220, 9227, 9388, 9481, 9487, 9515, 9582, 9689, 9729, 9776, 9781, 9824, 9828, 9842, 9881, 9913, 10055, 10098, 10100, 10140, 10159, 10223, 10237, 10294, 10438, 10501, 10546, 10671, 10756, 10790, 10972, 11005, 11012, 11084, 11120, 11153, 11300, 11363, 11877, 11961, 11980, 12882, 12967, 13001, 13183, 13216, 13223, 13295, 13331, 13364, 13511, 13574, 14088, 14172, 14191] .

    \end{Verbatim}

    \begin{Verbatim}[commandchars=\\\{\}]
{\color{incolor}In [{\color{incolor}34}]:} \PY{n}{df\PYZus{}hr\PYZus{}empleados\PYZus{}con\PYZus{}riesgo} \PY{o}{=} \PY{n}{df\PYZus{}hr}\PY{p}{[}\PY{n}{df\PYZus{}hr}\PY{o}{.}\PY{n}{index}\PY{o}{.}\PY{n}{isin}\PY{p}{(}\PY{n}{empleados\PYZus{}con\PYZus{}riesgo}\PY{p}{)}\PY{p}{]}
\end{Verbatim}


    \begin{Verbatim}[commandchars=\\\{\}]
{\color{incolor}In [{\color{incolor}35}]:} \PY{n}{df\PYZus{}hr\PYZus{}empleados\PYZus{}con\PYZus{}riesgo}\PY{p}{[}\PY{l+s+s1}{\PYZsq{}}\PY{l+s+s1}{prob\PYZus{}renuncia}\PY{l+s+s1}{\PYZsq{}}\PY{p}{]} \PY{o}{=} \PY{n}{df\PYZus{}empleados\PYZus{}con\PYZus{}riesgo}\PY{o}{.}\PY{n}{loc}\PY{p}{[}\PY{p}{:}\PY{p}{,}\PY{l+s+s1}{\PYZsq{}}\PY{l+s+s1}{prob\PYZus{}renuncia}\PY{l+s+s1}{\PYZsq{}}\PY{p}{]}
\end{Verbatim}


    \begin{Verbatim}[commandchars=\\\{\}]
C:\textbackslash{}ProgramData\textbackslash{}Anaconda3\textbackslash{}lib\textbackslash{}site-packages\textbackslash{}ipykernel\_launcher.py:1: SettingWithCopyWarning: 
A value is trying to be set on a copy of a slice from a DataFrame.
Try using .loc[row\_indexer,col\_indexer] = value instead

See the caveats in the documentation: http://pandas.pydata.org/pandas-docs/stable/indexing.html\#indexing-view-versus-copy
  """Entry point for launching an IPython kernel.

    \end{Verbatim}

    \begin{Verbatim}[commandchars=\\\{\}]
{\color{incolor}In [{\color{incolor}36}]:} \PY{n}{df\PYZus{}hr\PYZus{}empleados\PYZus{}con\PYZus{}riesgo} \PY{o}{=} \PY{n}{df\PYZus{}hr\PYZus{}empleados\PYZus{}con\PYZus{}riesgo}\PY{p}{[}\PY{o}{\PYZhy{}}\PY{l+m+mi}{1}\PY{p}{:}\PY{p}{]} \PY{o}{+} \PY{n}{df\PYZus{}hr\PYZus{}empleados\PYZus{}con\PYZus{}riesgo}\PY{p}{[}\PY{p}{:}\PY{o}{\PYZhy{}}\PY{l+m+mi}{1}\PY{p}{]}
\end{Verbatim}


    \begin{Verbatim}[commandchars=\\\{\}]
{\color{incolor}In [{\color{incolor}37}]:} \PY{n}{df\PYZus{}hr\PYZus{}empleados\PYZus{}con\PYZus{}riesgo}
\end{Verbatim}


\begin{Verbatim}[commandchars=\\\{\}]
{\color{outcolor}Out[{\color{outcolor}37}]:}        satisfaction\_level  last\_evaluation  number\_project  \textbackslash{}
         2011                  NaN              NaN             NaN   
         2094                  NaN              NaN             NaN   
         2206                  NaN              NaN             NaN   
         2214                  NaN              NaN             NaN   
         2410                  NaN              NaN             NaN   
         2415                  NaN              NaN             NaN   
         2421                  NaN              NaN             NaN   
         2475                  NaN              NaN             NaN   
         2498                  NaN              NaN             NaN   
         2550                  NaN              NaN             NaN   
         2570                  NaN              NaN             NaN   
         2703                  NaN              NaN             NaN   
         2723                  NaN              NaN             NaN   
         2801                  NaN              NaN             NaN   
         2842                  NaN              NaN             NaN   
         3038                  NaN              NaN             NaN   
         3056                  NaN              NaN             NaN   
         3079                  NaN              NaN             NaN   
         3106                  NaN              NaN             NaN   
         3114                  NaN              NaN             NaN   
         3160                  NaN              NaN             NaN   
         3326                  NaN              NaN             NaN   
         3345                  NaN              NaN             NaN   
         3518                  NaN              NaN             NaN   
         3710                  NaN              NaN             NaN   
         3780                  NaN              NaN             NaN   
         3890                  NaN              NaN             NaN   
         3921                  NaN              NaN             NaN   
         3989                  NaN              NaN             NaN   
         4076                  NaN              NaN             NaN   
         {\ldots}                   {\ldots}              {\ldots}             {\ldots}   
         10501                 NaN              NaN             NaN   
         10546                 NaN              NaN             NaN   
         10671                 NaN              NaN             NaN   
         10756                 NaN              NaN             NaN   
         10790                 NaN              NaN             NaN   
         10972                 NaN              NaN             NaN   
         11005                 NaN              NaN             NaN   
         11012                 NaN              NaN             NaN   
         11084                 NaN              NaN             NaN   
         11120                 NaN              NaN             NaN   
         11153                 NaN              NaN             NaN   
         11300                 NaN              NaN             NaN   
         11363                 NaN              NaN             NaN   
         11877                 NaN              NaN             NaN   
         11961                 NaN              NaN             NaN   
         11980                 NaN              NaN             NaN   
         12882                 NaN              NaN             NaN   
         12967                 NaN              NaN             NaN   
         13001                 NaN              NaN             NaN   
         13183                 NaN              NaN             NaN   
         13216                 NaN              NaN             NaN   
         13223                 NaN              NaN             NaN   
         13295                 NaN              NaN             NaN   
         13331                 NaN              NaN             NaN   
         13364                 NaN              NaN             NaN   
         13511                 NaN              NaN             NaN   
         13574                 NaN              NaN             NaN   
         14088                 NaN              NaN             NaN   
         14172                 NaN              NaN             NaN   
         14191                 NaN              NaN             NaN   
         
                average\_montly\_hours  time\_spend\_company  Work\_accident  left  \textbackslash{}
         2011                    NaN                 NaN            NaN   NaN   
         2094                    NaN                 NaN            NaN   NaN   
         2206                    NaN                 NaN            NaN   NaN   
         2214                    NaN                 NaN            NaN   NaN   
         2410                    NaN                 NaN            NaN   NaN   
         2415                    NaN                 NaN            NaN   NaN   
         2421                    NaN                 NaN            NaN   NaN   
         2475                    NaN                 NaN            NaN   NaN   
         2498                    NaN                 NaN            NaN   NaN   
         2550                    NaN                 NaN            NaN   NaN   
         2570                    NaN                 NaN            NaN   NaN   
         2703                    NaN                 NaN            NaN   NaN   
         2723                    NaN                 NaN            NaN   NaN   
         2801                    NaN                 NaN            NaN   NaN   
         2842                    NaN                 NaN            NaN   NaN   
         3038                    NaN                 NaN            NaN   NaN   
         3056                    NaN                 NaN            NaN   NaN   
         3079                    NaN                 NaN            NaN   NaN   
         3106                    NaN                 NaN            NaN   NaN   
         3114                    NaN                 NaN            NaN   NaN   
         3160                    NaN                 NaN            NaN   NaN   
         3326                    NaN                 NaN            NaN   NaN   
         3345                    NaN                 NaN            NaN   NaN   
         3518                    NaN                 NaN            NaN   NaN   
         3710                    NaN                 NaN            NaN   NaN   
         3780                    NaN                 NaN            NaN   NaN   
         3890                    NaN                 NaN            NaN   NaN   
         3921                    NaN                 NaN            NaN   NaN   
         3989                    NaN                 NaN            NaN   NaN   
         4076                    NaN                 NaN            NaN   NaN   
         {\ldots}                     {\ldots}                 {\ldots}            {\ldots}   {\ldots}   
         10501                   NaN                 NaN            NaN   NaN   
         10546                   NaN                 NaN            NaN   NaN   
         10671                   NaN                 NaN            NaN   NaN   
         10756                   NaN                 NaN            NaN   NaN   
         10790                   NaN                 NaN            NaN   NaN   
         10972                   NaN                 NaN            NaN   NaN   
         11005                   NaN                 NaN            NaN   NaN   
         11012                   NaN                 NaN            NaN   NaN   
         11084                   NaN                 NaN            NaN   NaN   
         11120                   NaN                 NaN            NaN   NaN   
         11153                   NaN                 NaN            NaN   NaN   
         11300                   NaN                 NaN            NaN   NaN   
         11363                   NaN                 NaN            NaN   NaN   
         11877                   NaN                 NaN            NaN   NaN   
         11961                   NaN                 NaN            NaN   NaN   
         11980                   NaN                 NaN            NaN   NaN   
         12882                   NaN                 NaN            NaN   NaN   
         12967                   NaN                 NaN            NaN   NaN   
         13001                   NaN                 NaN            NaN   NaN   
         13183                   NaN                 NaN            NaN   NaN   
         13216                   NaN                 NaN            NaN   NaN   
         13223                   NaN                 NaN            NaN   NaN   
         13295                   NaN                 NaN            NaN   NaN   
         13331                   NaN                 NaN            NaN   NaN   
         13364                   NaN                 NaN            NaN   NaN   
         13511                   NaN                 NaN            NaN   NaN   
         13574                   NaN                 NaN            NaN   NaN   
         14088                   NaN                 NaN            NaN   NaN   
         14172                   NaN                 NaN            NaN   NaN   
         14191                   NaN                 NaN            NaN   NaN   
         
                promotion\_last\_5years department  salary  prob\_renuncia  
         2011                     NaN        NaN     NaN            NaN  
         2094                     NaN        NaN     NaN            NaN  
         2206                     NaN        NaN     NaN            NaN  
         2214                     NaN        NaN     NaN            NaN  
         2410                     NaN        NaN     NaN            NaN  
         2415                     NaN        NaN     NaN            NaN  
         2421                     NaN        NaN     NaN            NaN  
         2475                     NaN        NaN     NaN            NaN  
         2498                     NaN        NaN     NaN            NaN  
         2550                     NaN        NaN     NaN            NaN  
         2570                     NaN        NaN     NaN            NaN  
         2703                     NaN        NaN     NaN            NaN  
         2723                     NaN        NaN     NaN            NaN  
         2801                     NaN        NaN     NaN            NaN  
         2842                     NaN        NaN     NaN            NaN  
         3038                     NaN        NaN     NaN            NaN  
         3056                     NaN        NaN     NaN            NaN  
         3079                     NaN        NaN     NaN            NaN  
         3106                     NaN        NaN     NaN            NaN  
         3114                     NaN        NaN     NaN            NaN  
         3160                     NaN        NaN     NaN            NaN  
         3326                     NaN        NaN     NaN            NaN  
         3345                     NaN        NaN     NaN            NaN  
         3518                     NaN        NaN     NaN            NaN  
         3710                     NaN        NaN     NaN            NaN  
         3780                     NaN        NaN     NaN            NaN  
         3890                     NaN        NaN     NaN            NaN  
         3921                     NaN        NaN     NaN            NaN  
         3989                     NaN        NaN     NaN            NaN  
         4076                     NaN        NaN     NaN            NaN  
         {\ldots}                      {\ldots}        {\ldots}     {\ldots}            {\ldots}  
         10501                    NaN        NaN     NaN            NaN  
         10546                    NaN        NaN     NaN            NaN  
         10671                    NaN        NaN     NaN            NaN  
         10756                    NaN        NaN     NaN            NaN  
         10790                    NaN        NaN     NaN            NaN  
         10972                    NaN        NaN     NaN            NaN  
         11005                    NaN        NaN     NaN            NaN  
         11012                    NaN        NaN     NaN            NaN  
         11084                    NaN        NaN     NaN            NaN  
         11120                    NaN        NaN     NaN            NaN  
         11153                    NaN        NaN     NaN            NaN  
         11300                    NaN        NaN     NaN            NaN  
         11363                    NaN        NaN     NaN            NaN  
         11877                    NaN        NaN     NaN            NaN  
         11961                    NaN        NaN     NaN            NaN  
         11980                    NaN        NaN     NaN            NaN  
         12882                    NaN        NaN     NaN            NaN  
         12967                    NaN        NaN     NaN            NaN  
         13001                    NaN        NaN     NaN            NaN  
         13183                    NaN        NaN     NaN            NaN  
         13216                    NaN        NaN     NaN            NaN  
         13223                    NaN        NaN     NaN            NaN  
         13295                    NaN        NaN     NaN            NaN  
         13331                    NaN        NaN     NaN            NaN  
         13364                    NaN        NaN     NaN            NaN  
         13511                    NaN        NaN     NaN            NaN  
         13574                    NaN        NaN     NaN            NaN  
         14088                    NaN        NaN     NaN            NaN  
         14172                    NaN        NaN     NaN            NaN  
         14191                    NaN        NaN     NaN            NaN  
         
         [189 rows x 11 columns]
\end{Verbatim}
            

    % Add a bibliography block to the postdoc
    
    
    
    \end{document}
